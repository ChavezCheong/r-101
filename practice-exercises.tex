\documentclass[]{article}
\usepackage{lmodern}
\usepackage{amssymb,amsmath}
\usepackage{ifxetex,ifluatex}
\usepackage{fixltx2e} % provides \textsubscript
\ifnum 0\ifxetex 1\fi\ifluatex 1\fi=0 % if pdftex
  \usepackage[T1]{fontenc}
  \usepackage[utf8]{inputenc}
\else % if luatex or xelatex
  \ifxetex
    \usepackage{mathspec}
  \else
    \usepackage{fontspec}
  \fi
  \defaultfontfeatures{Ligatures=TeX,Scale=MatchLowercase}
\fi
% use upquote if available, for straight quotes in verbatim environments
\IfFileExists{upquote.sty}{\usepackage{upquote}}{}
% use microtype if available
\IfFileExists{microtype.sty}{%
\usepackage{microtype}
\UseMicrotypeSet[protrusion]{basicmath} % disable protrusion for tt fonts
}{}
\usepackage[margin=1in]{geometry}
\usepackage{hyperref}
\hypersetup{unicode=true,
            pdftitle={Practice Exercises},
            pdfauthor={Chavez Cheong},
            pdfborder={0 0 0},
            breaklinks=true}
\urlstyle{same}  % don't use monospace font for urls
\usepackage{color}
\usepackage{fancyvrb}
\newcommand{\VerbBar}{|}
\newcommand{\VERB}{\Verb[commandchars=\\\{\}]}
\DefineVerbatimEnvironment{Highlighting}{Verbatim}{commandchars=\\\{\}}
% Add ',fontsize=\small' for more characters per line
\usepackage{framed}
\definecolor{shadecolor}{RGB}{248,248,248}
\newenvironment{Shaded}{\begin{snugshade}}{\end{snugshade}}
\newcommand{\AlertTok}[1]{\textcolor[rgb]{0.94,0.16,0.16}{#1}}
\newcommand{\AnnotationTok}[1]{\textcolor[rgb]{0.56,0.35,0.01}{\textbf{\textit{#1}}}}
\newcommand{\AttributeTok}[1]{\textcolor[rgb]{0.77,0.63,0.00}{#1}}
\newcommand{\BaseNTok}[1]{\textcolor[rgb]{0.00,0.00,0.81}{#1}}
\newcommand{\BuiltInTok}[1]{#1}
\newcommand{\CharTok}[1]{\textcolor[rgb]{0.31,0.60,0.02}{#1}}
\newcommand{\CommentTok}[1]{\textcolor[rgb]{0.56,0.35,0.01}{\textit{#1}}}
\newcommand{\CommentVarTok}[1]{\textcolor[rgb]{0.56,0.35,0.01}{\textbf{\textit{#1}}}}
\newcommand{\ConstantTok}[1]{\textcolor[rgb]{0.00,0.00,0.00}{#1}}
\newcommand{\ControlFlowTok}[1]{\textcolor[rgb]{0.13,0.29,0.53}{\textbf{#1}}}
\newcommand{\DataTypeTok}[1]{\textcolor[rgb]{0.13,0.29,0.53}{#1}}
\newcommand{\DecValTok}[1]{\textcolor[rgb]{0.00,0.00,0.81}{#1}}
\newcommand{\DocumentationTok}[1]{\textcolor[rgb]{0.56,0.35,0.01}{\textbf{\textit{#1}}}}
\newcommand{\ErrorTok}[1]{\textcolor[rgb]{0.64,0.00,0.00}{\textbf{#1}}}
\newcommand{\ExtensionTok}[1]{#1}
\newcommand{\FloatTok}[1]{\textcolor[rgb]{0.00,0.00,0.81}{#1}}
\newcommand{\FunctionTok}[1]{\textcolor[rgb]{0.00,0.00,0.00}{#1}}
\newcommand{\ImportTok}[1]{#1}
\newcommand{\InformationTok}[1]{\textcolor[rgb]{0.56,0.35,0.01}{\textbf{\textit{#1}}}}
\newcommand{\KeywordTok}[1]{\textcolor[rgb]{0.13,0.29,0.53}{\textbf{#1}}}
\newcommand{\NormalTok}[1]{#1}
\newcommand{\OperatorTok}[1]{\textcolor[rgb]{0.81,0.36,0.00}{\textbf{#1}}}
\newcommand{\OtherTok}[1]{\textcolor[rgb]{0.56,0.35,0.01}{#1}}
\newcommand{\PreprocessorTok}[1]{\textcolor[rgb]{0.56,0.35,0.01}{\textit{#1}}}
\newcommand{\RegionMarkerTok}[1]{#1}
\newcommand{\SpecialCharTok}[1]{\textcolor[rgb]{0.00,0.00,0.00}{#1}}
\newcommand{\SpecialStringTok}[1]{\textcolor[rgb]{0.31,0.60,0.02}{#1}}
\newcommand{\StringTok}[1]{\textcolor[rgb]{0.31,0.60,0.02}{#1}}
\newcommand{\VariableTok}[1]{\textcolor[rgb]{0.00,0.00,0.00}{#1}}
\newcommand{\VerbatimStringTok}[1]{\textcolor[rgb]{0.31,0.60,0.02}{#1}}
\newcommand{\WarningTok}[1]{\textcolor[rgb]{0.56,0.35,0.01}{\textbf{\textit{#1}}}}
\usepackage{graphicx,grffile}
\makeatletter
\def\maxwidth{\ifdim\Gin@nat@width>\linewidth\linewidth\else\Gin@nat@width\fi}
\def\maxheight{\ifdim\Gin@nat@height>\textheight\textheight\else\Gin@nat@height\fi}
\makeatother
% Scale images if necessary, so that they will not overflow the page
% margins by default, and it is still possible to overwrite the defaults
% using explicit options in \includegraphics[width, height, ...]{}
\setkeys{Gin}{width=\maxwidth,height=\maxheight,keepaspectratio}
\IfFileExists{parskip.sty}{%
\usepackage{parskip}
}{% else
\setlength{\parindent}{0pt}
\setlength{\parskip}{6pt plus 2pt minus 1pt}
}
\setlength{\emergencystretch}{3em}  % prevent overfull lines
\providecommand{\tightlist}{%
  \setlength{\itemsep}{0pt}\setlength{\parskip}{0pt}}
\setcounter{secnumdepth}{0}
% Redefines (sub)paragraphs to behave more like sections
\ifx\paragraph\undefined\else
\let\oldparagraph\paragraph
\renewcommand{\paragraph}[1]{\oldparagraph{#1}\mbox{}}
\fi
\ifx\subparagraph\undefined\else
\let\oldsubparagraph\subparagraph
\renewcommand{\subparagraph}[1]{\oldsubparagraph{#1}\mbox{}}
\fi

%%% Use protect on footnotes to avoid problems with footnotes in titles
\let\rmarkdownfootnote\footnote%
\def\footnote{\protect\rmarkdownfootnote}

%%% Change title format to be more compact
\usepackage{titling}

% Create subtitle command for use in maketitle
\providecommand{\subtitle}[1]{
  \posttitle{
    \begin{center}\large#1\end{center}
    }
}

\setlength{\droptitle}{-2em}

  \title{Practice Exercises}
    \pretitle{\vspace{\droptitle}\centering\huge}
  \posttitle{\par}
    \author{Chavez Cheong}
    \preauthor{\centering\large\emph}
  \postauthor{\par}
      \predate{\centering\large\emph}
  \postdate{\par}
    \date{3/3/2021}


\begin{document}
\maketitle

\begin{Shaded}
\begin{Highlighting}[]
\CommentTok{\#install.packages("tidyverse")}
\CommentTok{\#install.packages("dplyr")}
\CommentTok{\#install.packages("ggplot2")}
\end{Highlighting}
\end{Shaded}

\begin{Shaded}
\begin{Highlighting}[]
\FunctionTok{library}\NormalTok{(tidyverse)}
\FunctionTok{library}\NormalTok{(dplyr)}
\FunctionTok{library}\NormalTok{(ggplot2)}
\end{Highlighting}
\end{Shaded}

Today we'll be using a dataset about mammalian sleep that comes built in
to ggplot2.

\begin{Shaded}
\begin{Highlighting}[]
\NormalTok{msleep}
\end{Highlighting}
\end{Shaded}

\begin{verbatim}
## # A tibble: 83 x 11
##    name  genus vore  order conservation sleep_total sleep_rem sleep_cycle
##    <chr> <chr> <chr> <chr> <chr>              <dbl>     <dbl>       <dbl>
##  1 Chee~ Acin~ carni Carn~ lc                  12.1      NA        NA    
##  2 Owl ~ Aotus omni  Prim~ <NA>                17         1.8      NA    
##  3 Moun~ Aplo~ herbi Rode~ nt                  14.4       2.4      NA    
##  4 Grea~ Blar~ omni  Sori~ lc                  14.9       2.3       0.133
##  5 Cow   Bos   herbi Arti~ domesticated         4         0.7       0.667
##  6 Thre~ Brad~ herbi Pilo~ <NA>                14.4       2.2       0.767
##  7 Nort~ Call~ carni Carn~ vu                   8.7       1.4       0.383
##  8 Vesp~ Calo~ <NA>  Rode~ <NA>                 7        NA        NA    
##  9 Dog   Canis carni Carn~ domesticated        10.1       2.9       0.333
## 10 Roe ~ Capr~ herbi Arti~ lc                   3        NA        NA    
## # ... with 73 more rows, and 3 more variables: awake <dbl>, brainwt <dbl>,
## #   bodywt <dbl>
\end{verbatim}

You usually won't be working with data that's already built into R, so
for practice we're going to read in the same \texttt{msleep} data as a
CSV file.

\begin{Shaded}
\begin{Highlighting}[]
\NormalTok{mammal }\OtherTok{\textless{}{-}} \FunctionTok{read\_csv}\NormalTok{(}\StringTok{"data/mammal{-}sleep.csv"}\NormalTok{)}
\end{Highlighting}
\end{Shaded}

\hypertarget{exercise-1}{%
\subsubsection{Exercise 1}\label{exercise-1}}

\textbf{What are the dimensions of \texttt{mammal()}? How many rows and
columns does it have?}

Type \texttt{head(mammal)} in the Console window. What does this do? How
about \texttt{tail(mammal)}?

\hypertarget{exercise-2}{%
\subsubsection{Exercise 2}\label{exercise-2}}

Let's do some exploratory data analysis and look at summary statistics!

You've learned the assignment operator, so it's time to introduce the
piping operator \texttt{\%\textgreater{}\%}. This is used to pass the
output of one line to the next, like in the example below.

\begin{Shaded}
\begin{Highlighting}[]
\NormalTok{mammal }\SpecialCharTok{\%\textgreater{}\%}
  \FunctionTok{filter}\NormalTok{(order }\SpecialCharTok{==} \StringTok{"Primates"}\NormalTok{) }\SpecialCharTok{\%\textgreater{}\%}
  \FunctionTok{select}\NormalTok{(name, sleep\_total) }\SpecialCharTok{\%\textgreater{}\%}
  \FunctionTok{arrange}\NormalTok{(}\FunctionTok{desc}\NormalTok{(sleep\_total)) }\SpecialCharTok{\%\textgreater{}\%}
  \FunctionTok{head}\NormalTok{(}\DecValTok{5}\NormalTok{)}
\end{Highlighting}
\end{Shaded}

\begin{verbatim}
## # A tibble: 5 x 2
##   name         sleep_total
##   <chr>              <dbl>
## 1 Owl monkey          17  
## 2 Slow loris          11  
## 3 Potto               11  
## 4 Patas monkey        10.9
## 5 Macaque             10.1
\end{verbatim}

\textbf{Try this on your own -- see if you can find the mammal of class
Rodentia that spends the least time awake.}

R can make logical comparisons like the one in \texttt{filter()} above.
Use \texttt{!=} to mean `does not equal'. You can also use operators
like \texttt{\textgreater{}} or \texttt{\textless{}=} to mean `is
greater than' and `is less than or equal to'. Also \texttt{is.na()} and
\texttt{!is.na()}. Remember to use \texttt{==} for equality, not
\texttt{=}. Use \texttt{\textbar{}} as ``or'' and \texttt{\&} as ``and''
for multiple conditions. (Also \texttt{\%in\%}!)

\hypertarget{exercise-3}{%
\subsubsection{Exercise 3}\label{exercise-3}}

If you want to look at the contents of one column (or vector) in the
data, you can subset a dataframe like so:

\begin{Shaded}
\begin{Highlighting}[]
\NormalTok{mammal}\SpecialCharTok{$}\NormalTok{sleep\_total}
\end{Highlighting}
\end{Shaded}

\begin{verbatim}
##  [1] 12.1 17.0 14.4 14.9  4.0 14.4  8.7  7.0 10.1  3.0  5.3  9.4 10.0 12.5
## [15] 10.3  8.3  9.1 17.4  5.3 18.0  3.9 19.7  2.9  3.1 10.1 10.9 14.9 12.5
## [29]  9.8  1.9  2.7  6.2  6.3  8.0  9.5  3.3 19.4 10.1 14.2 14.3 12.8 12.5
## [43] 19.9 14.6 11.0  7.7 14.5  8.4  3.8  9.7 15.8 10.4 13.5  9.4 10.3 11.0
## [57] 11.5 13.7  3.5  5.6 11.1 18.1  5.4 13.0  8.7  9.6  8.4 11.3 10.6 16.6
## [71] 13.8 15.9 12.8  9.1  8.6 15.8  4.4 15.6  8.9  5.2  6.3 12.5  9.8
\end{verbatim}

\textbf{Try finding the average total amount of sleep of mammals in the
data on your own.}

The \texttt{summary()} function can provide good information.

\begin{Shaded}
\begin{Highlighting}[]
\FunctionTok{summary}\NormalTok{(mammal)}
\end{Highlighting}
\end{Shaded}

\begin{verbatim}
##      name              genus               vore          
##  Length:83          Length:83          Length:83         
##  Class :character   Class :character   Class :character  
##  Mode  :character   Mode  :character   Mode  :character  
##                                                          
##                                                          
##                                                          
##                                                          
##     order           conservation        sleep_total      sleep_rem    
##  Length:83          Length:83          Min.   : 1.90   Min.   :0.100  
##  Class :character   Class :character   1st Qu.: 7.85   1st Qu.:0.900  
##  Mode  :character   Mode  :character   Median :10.10   Median :1.500  
##                                        Mean   :10.43   Mean   :1.875  
##                                        3rd Qu.:13.75   3rd Qu.:2.400  
##                                        Max.   :19.90   Max.   :6.600  
##                                                        NA's   :22     
##   sleep_cycle         awake          brainwt            bodywt        
##  Min.   :0.1167   Min.   : 4.10   Min.   :0.00014   Min.   :   0.005  
##  1st Qu.:0.1833   1st Qu.:10.25   1st Qu.:0.00290   1st Qu.:   0.174  
##  Median :0.3333   Median :13.90   Median :0.01240   Median :   1.670  
##  Mean   :0.4396   Mean   :13.57   Mean   :0.28158   Mean   : 166.136  
##  3rd Qu.:0.5792   3rd Qu.:16.15   3rd Qu.:0.12550   3rd Qu.:  41.750  
##  Max.   :1.5000   Max.   :22.10   Max.   :5.71200   Max.   :6654.000  
##  NA's   :51                       NA's   :27
\end{verbatim}

Does it make sense to use the summary function on the variable
\texttt{order} in our data?

Reminder: Now you've made a few more edits to your R Markdown files, try
committing your changes!

\hypertarget{exercise-4}{%
\subsubsection{Exercise 4}\label{exercise-4}}

The \texttt{mutate} function allows you to create new variables. Let's
create a variable with body weight in pounds instead of kilograms.

\begin{Shaded}
\begin{Highlighting}[]
\NormalTok{mammal }\OtherTok{\textless{}{-}}\NormalTok{ mammal }\SpecialCharTok{\%\textgreater{}\%}
  \FunctionTok{mutate}\NormalTok{(}\AttributeTok{bodywt\_lbs =} \FunctionTok{round}\NormalTok{(bodywt }\SpecialCharTok{*} \FloatTok{2.205}\NormalTok{, }\DecValTok{2}\NormalTok{))}
\end{Highlighting}
\end{Shaded}

Now let's use \texttt{case\_when()} within \texttt{mutate()} to create a
variable that shows whether an animal spends more time asleep than
awake.

\begin{Shaded}
\begin{Highlighting}[]
\NormalTok{mammal }\OtherTok{\textless{}{-}}\NormalTok{ mammal }\SpecialCharTok{\%\textgreater{}\%}
  \FunctionTok{mutate}\NormalTok{(}\AttributeTok{sleepy =} \FunctionTok{case\_when}\NormalTok{(}
\NormalTok{    sleep\_total }\SpecialCharTok{\textgreater{}}\NormalTok{ awake }\SpecialCharTok{\textasciitilde{}} \ConstantTok{TRUE}\NormalTok{,}
\NormalTok{    sleep\_total }\SpecialCharTok{\textless{}=}\NormalTok{ awake }\SpecialCharTok{\textasciitilde{}} \ConstantTok{FALSE}
\NormalTok{  ))}
\end{Highlighting}
\end{Shaded}

\textbf{On your own, create a variable containing the number of sleep
cycles the animals experience each night (hint: divide the total amount
of sleep by the length of the sleep cycle). Round to one decimal place
and call the new variable \texttt{num\_cycles}}

A syntax tip for case\_when: use
\texttt{TRUE\ \textasciitilde{}\ orig\_variable\_name} as a kind of
``else'' or ``in all other cases''. So we could also use the following
code to create \texttt{sleepy}.

\begin{Shaded}
\begin{Highlighting}[]
\NormalTok{mammal }\OtherTok{\textless{}{-}}\NormalTok{ mammal }\SpecialCharTok{\%\textgreater{}\%}
  \FunctionTok{mutate}\NormalTok{(}\AttributeTok{sleepy =} \FunctionTok{case\_when}\NormalTok{(}
\NormalTok{    sleep\_total }\SpecialCharTok{\textgreater{}}\NormalTok{ awake }\SpecialCharTok{\textasciitilde{}} \ConstantTok{TRUE}\NormalTok{,}
    \ConstantTok{TRUE} \SpecialCharTok{\textasciitilde{}} \ConstantTok{FALSE}
\NormalTok{  ))}
\end{Highlighting}
\end{Shaded}

Every value of a variable must be of the same type (i.e.~character,
numeric, integer, etc.) -- be careful to only create values of the same
type when using \texttt{case\_when()} otherwise your mutate won't work
(especially with NA values). \texttt{as.numeric()},
\texttt{as.character()}, \texttt{as.integer()} are useful functions to
coerce NA values (and others) to your desired variable type.

\hypertarget{exercise-5}{%
\subsubsection{Exercise 5}\label{exercise-5}}

\texttt{group\_by()} can be very helpful if you want to answer questions
about specific groups or categories of observations in the data.

For example, we can use \texttt{group\_by()} to determine how many
animals spend more time asleep than awake.

\begin{Shaded}
\begin{Highlighting}[]
\NormalTok{mammal }\SpecialCharTok{\%\textgreater{}\%}
  \FunctionTok{group\_by}\NormalTok{(sleepy) }\SpecialCharTok{\%\textgreater{}\%}
  \FunctionTok{count}\NormalTok{()}
\end{Highlighting}
\end{Shaded}

\begin{verbatim}
## # A tibble: 2 x 2
## # Groups:   sleepy [2]
##   sleepy     n
##   <lgl>  <int>
## 1 FALSE     52
## 2 TRUE      31
\end{verbatim}

We can also use \texttt{summarize()} to calculate proportions, means, or
other descriptions of our grouped data.

\begin{Shaded}
\begin{Highlighting}[]
\NormalTok{mammal }\SpecialCharTok{\%\textgreater{}\%}
  \FunctionTok{group\_by}\NormalTok{(sleepy) }\SpecialCharTok{\%\textgreater{}\%}
  \FunctionTok{summarize}\NormalTok{(}\AttributeTok{prop =} \FunctionTok{n}\NormalTok{()}\SpecialCharTok{/}\FunctionTok{nrow}\NormalTok{(mammal))}
\end{Highlighting}
\end{Shaded}

\begin{verbatim}
## # A tibble: 2 x 2
##   sleepy  prop
##   <lgl>  <dbl>
## 1 FALSE  0.627
## 2 TRUE   0.373
\end{verbatim}

\textbf{Use group\_by() to determine the proportion of mammals in the
data that are carnivores.}

\hypertarget{exercise-6}{%
\subsubsection{Exercise 6}\label{exercise-6}}

\texttt{ggplot2} is not the only package for making plots in R, but it's
a really good place to start.

You can find a great ggplot2 cheatsheet
\href{https://rstudio.com/wp-content/uploads/2016/11/ggplot2-cheatsheet-2.1.pdf}{here.}

Let's create a plot to look at how total sleep and REM sleep are
related.

\begin{Shaded}
\begin{Highlighting}[]
\FunctionTok{ggplot}\NormalTok{(}\AttributeTok{data =}\NormalTok{ mammal, }\AttributeTok{mapping =} \FunctionTok{aes}\NormalTok{(}\AttributeTok{x =}\NormalTok{ sleep\_total, }\AttributeTok{y =}\NormalTok{ sleep\_rem, }\AttributeTok{color =}\NormalTok{ sleepy)) }\SpecialCharTok{+}
  \FunctionTok{geom\_point}\NormalTok{() }\SpecialCharTok{+}
  \FunctionTok{theme\_light}\NormalTok{() }\SpecialCharTok{+}
  \FunctionTok{labs}\NormalTok{(}\AttributeTok{title =} \StringTok{"REM Sleep vs. Total Sleep in Mammals"}\NormalTok{,}
       \AttributeTok{subtitle =} \StringTok{"from msleep, a dataset of the ggplot2 package"}\NormalTok{,}
       \AttributeTok{x =} \StringTok{"Total Sleep (hrs)"}\NormalTok{,}
       \AttributeTok{y =} \StringTok{"REM Sleep (hrs)"}\NormalTok{, }
       \AttributeTok{color =} \StringTok{"More Time Asleep than Awake"}\NormalTok{)}
\end{Highlighting}
\end{Shaded}

\begin{verbatim}
## Warning: Removed 22 rows containing missing values (geom_point).
\end{verbatim}

\includegraphics{practice-exercises_files/figure-latex/unnamed-chunk-10-1.pdf}

You may need to filter out NAs before plotting -- R does it
automatically here.

\begin{itemize}
\tightlist
\item
  data is always the first argument (can go over piping operator
  exception)
\item
  within \texttt{aes()} (short for aesthetics), assign variables to the
  x/y axes and other elements of the plot
\item
  use another layer to specify the kind of plot --
  \texttt{geom\_point()}, \texttt{geom\_histogram()}, and
  \texttt{geom\_boxplot()} are a few common examples
\item
  add a theme to manipulate the appearance of your plot --
  \texttt{theme\_light()} and \texttt{theme\_classic()} are two of many
  options
\item
  use \texttt{labs()} to add labels
\item
  tip: use \texttt{alpha()} for transparency on crowded scatterplots
\end{itemize}

Notes on \texttt{theme()}: You can build highly cutomized plots by
formatting the non-data elements with arguments within \texttt{theme()}.
See \href{https://ggplot2.tidyverse.org/reference/theme.html}{this link}
for a list of optional arguments. This is great for adding
horizontal/vertical lines and other annotations to tell the story of
your data.

Now let's create a boxplot -- these are helpful for visualizing
categorical variables.

\begin{Shaded}
\begin{Highlighting}[]
\FunctionTok{ggplot}\NormalTok{(}\AttributeTok{data =}\NormalTok{ mammal, }\FunctionTok{aes}\NormalTok{(}\AttributeTok{x =}\NormalTok{ vore, }\AttributeTok{y =}\NormalTok{ sleep\_rem)) }\SpecialCharTok{+}
  \FunctionTok{geom\_boxplot}\NormalTok{() }\SpecialCharTok{+}
  \FunctionTok{labs}\NormalTok{(}\AttributeTok{title =} \StringTok{"REM Sleep Across Mammalian Diets"}\NormalTok{, }
       \AttributeTok{y =} \StringTok{"REM Sleep (hrs)"}\NormalTok{,}
       \AttributeTok{x =} \StringTok{"{-}vore Classification"}\NormalTok{)}
\end{Highlighting}
\end{Shaded}

\begin{verbatim}
## Warning: Removed 22 rows containing non-finite values (stat_boxplot).
\end{verbatim}

\includegraphics{practice-exercises_files/figure-latex/unnamed-chunk-11-1.pdf}

Note that we didn't add a theme here -- this is the default appearance
for plots using ggplot2.

\textbf{Now play with ggplot2 and the other skills you've learned to
create a plot that interests you.}


\end{document}
